\documentclass[12pt]{article}
\usepackage[utf8]{inputenc}
\usepackage{physics}
\usepackage{graphicx}
\usepackage{geometry}
\usepackage{cleveref}
\usepackage{cite,natbib}
\usepackage{xcolor}
\usepackage{multirow}
\usepackage{amsmath}
\usepackage{array}
\usepackage{longtable}
\usepackage{mathptmx}

\geometry{a4paper, portrait, margin=1in}
\renewcommand{\arraystretch}{2} %more spacing between rows in tabular
\usepackage{bm}

\begin{document}
\section*{S5. Summary tables of the methods used for the estimation of "fracture energy"}

In this supplemental document, we decided to report the methods for the estimation of "fracture energy" keeping when possible the original equations and symbols as in the reported reference.

\section{Geology}
\begin{longtable}{|p{3.5 cm}|p{6.5 cm}|p{1.25 cm}|p{1.25 cm}|p{1.5 cm}|}
\hline
\textbf{Reference} & \textbf{Original definition} & \textbf{Unit} & \textbf{Note} & \textbf{Name in repository} \\ \hline
Chester et al., 2005 & \( \displaystyle 5 S_{UC} \gamma \), with $\gamma$ = 1 $J/m^2$ (ultracataclasite) & $J/m^2$ & PD & global surface energy \\
Chester et al., 2005 & \( \displaystyle 5(S_{SF}+S_{MF}+S_{UC}) \gamma \), with $\gamma$ = 1 $J/m^2$ (ultracataclasite and damage zone) & $J/m^2$ & PD, FD & global surface energy \\
Chester et al., 2005 & total slip of Punchbowl fault & m & & global slip \\
Chester et al., 2005 & global energy (ultracataclasite and damage zone) divided by 10000 events, upper and lower bound assuming during one earthquake refracturing of 50\% healed clasts requires 1/2 $\gamma$ & $J/m^2$ & & surface energy per event \\
Chester et al., 2005 & global slip divided by 10000 events & m & & slip per event \\
Ma et al., 2006 & \( \displaystyle  G_{MSZ} = S_{MSZ} \lambda G_c \), with $\lambda$ = 6.6 and $G_c$ = 1 $J/m^2$ & $J/m^2$ & PD & global surface energy \\
Ma et al., 2006 & 1999 Chi-Chi earthquake slip multiplied by 6.6 events & m & & global slip \\
Ma et al., 2006 & global surface energy divided by 6.6 events (in 2 cm MSZ) & $J/m^2$ & PD & surface energy per event \\
Ma et al., 2006 & 1999 Chi-Chi earthquake slip & m & & slip per event \\
Ma et al., 2006 & \( \displaystyle W_b \) & $J/m^2$ & & not used \\
Wilson et al., 2005 & unknown (San Andreas Fault at Tejon Pass) & $J/m^2$ & * & global surface energy \\
Wilson et al., 2005 & Total displacement (San Andreas Fault at Tejon Pass) & m & & global slip \\
Wilson et al., 2005 & Assuming each earthquake generated a gouge zone 10  mm thick. Surface energy $=$ 10 multiplied by 80 $m^2/g$ multiplied by the specific surface energy of quartz of 1 to 1.8 $J/m^{2}$ (San Andreas Fault at Tejon Pass). & $J/m^2$ & PD, BET & surface energy per event \\
Wilson et al., 2005 & global slip divided by 7000 or 10000 events: total thickness 70 to 100 m divided by 10 mm, the thickness of each event (San Andreas Fault at Tejon Pass) & m & & slip per event \\
Wilson et al., 2005 & 10-30 fractures 1 mm thick times the energy per 1 mm thick gouge: 80 $m^2/g$ times specific surface energy of quartz of 1 to 1.8 $J/m^2$ (Bosman Fault) & $J/m^2$ & PD & surface energy per event \\
Wilson et al., 2005 & Slip of 1997 Hartebeestfontein gold mine earthquake (Bosman Fault) & m & & slip per event \\
Wilson et al., 2005 & \( \displaystyle U_f = \tau d\) & $J/m^2$ & & not used \\
Pittarello et al., 2008 & \( \displaystyle U_s = (A_{SZ} + A_{DZ}) \gamma \) & $J/m^2$ & PD in PST grains & surface energy per event \\
Pittarello et al., 2008 & \( \displaystyle Q = [H(1-\phi) + c_p (T_m - T_{hr})] \rho w   \) & $J/m^2$ & & not used \\
Pittarello et al., 2008 & separation aplitic dyke crosscut by the fault & m & & slip per event \\
Reches \& Dewers, 2005 & see Wilson et al., 2005 & $J/m^2$ & & surface energy per event \\
Reches \& Dewers, 2005 & see Wilson et al., 2005 & m & PD & slip per event \\
Johnson et al., 2021 & lower bound: \( \displaystyle U_s=\frac{1}{2} \left ( U_s^{garnet} V^{garnet} \right )\) with $V^{garnet}=$0.01-0.05; upper bound: \( \displaystyle U_s=\frac{1}{6} \left ( U_s^{garnet} V^{garnet} \right ) \) with $V^{garnet}=$1, $U_s^{garnet} = 3.551\,J/m^2$, $U_{sa}$ is $U_s$ integrated over fault width & $J/m^2$ & FD, PD & surface energy per event \\
Johnson et al., 2021 & unknown & m & * & slip per event \\
Olgaard \& Brace, 1983 & \( \displaystyle W_A = S \gamma\) & $J/m^3$ & XRS, BET & not used \\
Olgaard \& Brace, 1983 & $W_A$ multiplied by fault thickness 0.01 or 0.05 m & $J/m^2$ & & surface energy per event \\
Olgaard \& Brace, 1983 & average fault offset & m & & slip per event \\

\hline
\end{longtable}
PD: particle density, FD: fracture density, BET: Brunauer, Emmett and Teller, PST: pseudotachylyte, XRS: X-Ray sedimentation.
*: not available in the original reference.

\section{Laboratory}

\begin{longtable}{|p{3.5 cm}|p{6.5 cm}|p{1.25 cm}|p{1.25 cm}|p{1.5 cm}|}
\hline
\textbf{Reference} & \textbf{Original definition} & \textbf{Unit} & \textbf{Note} & \textbf{Name in repository} \\ \hline
Aben et al., 2020b & \(\displaystyle W_b = G_{nom} \) from Wong, 1986 & & & $=$ \\
Aben et al., 2020b & \(\displaystyle \Gamma_{off} = 2 \sum_{i=1}^{n} (\rho_{i}^{frac}-\rho_{0}^{frac}) x_i \Gamma_i \) & & CD & $=$ \\
Aben et al., 2019 & \( \displaystyle G_{off} \approx \frac{1}{2} w \overline{\sigma}_{ij} \Delta S_{ijkl} \overline{\sigma}_{kl} \) & & AWT & $=$ \\
Aben et al., 2019 & \( \displaystyle G_c=G_{nom} \) from Wong, 1986 & & & $=$ \\
Beeler et al., 2012 & \( \displaystyle G_e = \frac{1}{2} (\tau_{yield}-\tau_{sliding}) d_* \) & & & $=$ \\
Beeler et al., 2012 & \( \displaystyle G_e^{min} = \frac{1}{2} k d_*^2 \) & $J/m^2$ & & $=$ \\
Lockner et al., 1991 & \( \displaystyle G_t = E_f + G \) & & & $=$ \\
Lockner et al., 1991 & \( \displaystyle G_c = G_{nom} \) from Wong, 1986 & & & $=$ \\
Lockner et al., 1991 & \( \displaystyle E_f = \tau_f \delta^* \)& & & $=$ \\
Moore \& Lockner, 1995 & \( \displaystyle G_{II} \) from Wong, 1986 & & & $=$ \\
Okubo \& Dieterich, 1981 & \( \displaystyle G = \frac{1}{2} (\tau_p - \tau_r) d_r \) & & & $=$ \\
Okubo \& Dieterich, 1984 & \( \displaystyle G = \frac{1}{2} (\tau_p - \tau_r) d_r \) & & & $=$ \\
Svetlizky \& Fineberg, 2014 & \( \displaystyle G = \frac{1-\nu^2}{E} f_{II}(C_f) K_{II}^2 = \Gamma \) & & & Gamma \\
Bayart et al., 2016a & \( \displaystyle G = \frac{1-\nu^2}{E} f_{II}(C_f) K_{II}^2 = \Gamma \) & & & Gamma \\
Bayart et al., 2018 & \( \displaystyle G = \frac{1-\nu^2}{E} f_{II}(C_f) K_{II}^2 = \Gamma \) & & & Gamma \\
Bayart et al., 2018 & \( \displaystyle G_{stat} = \Gamma \rightarrow K_{II}^{stat} = \sqrt{4 \mu (1-k^2) \Gamma} = K_c \) & & & not used \\
Kammer \& McLaskey, 2019 & \( \displaystyle \Gamma = G_{II} = \frac{1-\nu^2}{E} K_{II}^2 f_{II}(C_f)\) & & & Gamma \\
Ke et al., 2018 & \( \displaystyle G_{II}^d = \Gamma \) & & & Gamma \\
Ke et al., 2018 & \( \displaystyle G_{II}^s = \frac{\alpha}{E_0} (K_{II}^s)^2 \) & & & not used \\
Fukuyama \& Mizoguchi, 2010 & \( \displaystyle E_F = \int_{0}^{D_c} S(\Delta u) - S(D_c) \,d \Delta u \) & & & $=$ \\
Violay et al., 2014a & "Fracture energy" = integral from slip = 0 (or Da?) to an arbitrary slip value (0.1 and 1 m) & & $**$ & $=$ \\
Chang et al., 2012 & \( \displaystyle E_C = 0.605 \cdot D^{0.933} \) & & & not used \\
Chang et al., 2012 & \( \displaystyle E_T = E_C \sigma_n \approx \tau D\) & & & $=$ \\
Passelegue et al., 2016a & \( \displaystyle E_g = \frac{1}{2} (\tau_p - \tau_r) D_c \) & $J/m^2$ & & $=$ \\
Passelegue et al., 2016a & \( \displaystyle E_h = \tau_r D \) & & & not used \\
Passelegue et al., 2016a & \( \displaystyle E_r = \left ( \frac{1}{2} (\tau_p - \tau_f) + (\tau_f - \tau_r) \right ) D - E_g  \) & & & not used \\
Lockner et al., 2017 & \( \displaystyle W_T = \frac{1}{2} (\tau_p - \tau_e) \Delta x \) & & & $=$ \\
Harbord et al., 2021 & \( \displaystyle W_b = \int_{0}^{D_{min}} [\tau(D) - \tau_{min}] \,d D \) & & & $=$ \\
Harbord et al., 2021 & \( \displaystyle W_r = \int_{D_{min}}^{D_{tot}} [\tau(D) - \tau_{min}] \,d D \) & & & not used \\
Passelegue et al., 2016b & \( \displaystyle E_g = \int_{0}^{D_{ss}} (\tau_{(\Delta D)} - \tau_{ss}) \,d D \) & & $*$ & not used \\
Paglialunga et al., 2022 & \( \displaystyle W_{b,tip} = \int_{0}^{D_c} (\tau(D)-\tau(D_c)) \,d D \) & & & $=$ \\
Paglialunga et al., 2022 & \( \displaystyle W_b = \int_{0}^{D_m} (\tau(D)-\tau(D_m)) \,d D \) & & & $=$ \\
Paglialunga et al., 2022 & \( \displaystyle G_c = \frac{1-\nu^2}{E} K_{II}^2 (C_f) f_{II} (C_f)\) & & LEFM & not used \\
Paglialunga et al., 2022 & \( \displaystyle G_c = \frac{16(1-\nu)}{9\pi} \frac{(\tau_p-\tau_r)^2}{\mu} x_c f_{II} (C_f) \) & & CZM & not used \\
Cornelio et al., 2020 & \( \displaystyle W_b = \int_{U_{in}}^{U_{min}} (\tau(U) - \tau_{min}) \,d U \) & & & $=$ \\
Cornelio et al., 2020 & \( \displaystyle W_r \) & & & not used \\
Cornelio et al., 2019 & \( \displaystyle G_c = \int_{0}^{D_c} (\tau(u) - \tau_{min}) \,d u \) & & & $=$ \\
Passelegue et al., 2017 & \( \displaystyle G = \frac{1}{2} (\tau_s - \tau_d) D_c \) & $J/m^2$ & & $=$ \\
Wong, 1986 & \( \displaystyle G = \int_{0}^{\infty} (\tau - \tau^f) \,d \delta = [(A-a)\sigma_n + (B-b)] \int_{0}^{\infty} f(\delta) h(\delta) \,d \delta \) & $J/m^2$ & $\sigma_n = const $ & $=$ \\
Wong, 1986 & \( \displaystyle G_{nom}(\tau_p) = \int_{0}^{\infty} (\tau-\tau^f) \,d\delta = (\tau^p-\tau^f) \int_{0}^{\infty} h(\tau^p,\delta) \,d\delta \) & $J/m^2$ & triaxial loading & not used \\
Wong, 1986 & \( \displaystyle \langle \delta_{nom} \rangle = \int_{0}^{\infty} h(\tau^p,\delta) \,d\delta \) & & & $=$ \\
Wong, 1982 & \( \displaystyle W = \int_{C'}^{C} \sigma(d \epsilon_1 - d e_1) - P D \) & $J/m^3$ & & not used \\
Wong, 1982 & \( \displaystyle \underline{G} = \int_{0}^{\Delta u^*} [\tau(\Delta u) -\tau^f] \,d(\Delta u) = (\tau^p-\tau^f) \overline{\Delta u} \) & $J/m^2$ & & G \\
Wong, 1982 & \( \displaystyle M = 2 \Gamma (S_V - S_V^0) \) & $J/m^3$ & CD & not used \\
Wong, 1982 & \( \displaystyle M^1 = 2 \Gamma (S_V^p - S_V^f) \cdot w \) & $J/m^2$ & CD & not used \\
Liu \& Rummel, 1990 & \( \displaystyle G = \int_{0}^{\delta^*} (\tau - \tau_f) \,d\delta = [(A-a)\delta_n + (B-b)] \int_{0}^{\delta^*} f(\delta) h(\delta) \,d\delta \) & & $\sigma_n = const$ & $=$ \\
Liu \& Rummel, 1990 & \( \displaystyle G_{nom}(\tau_p) = \int_{0}^{\delta^*} (\tau - \tau_f)_{nom} \,d\delta = (\tau_p-\tau_f)_{nom} \int_{0}^{\delta^*} h(\tau_p,\delta) \,d\delta \) & & triaxial loading & not used \\
Hakami \& Stephansson, 1990 & \( \displaystyle G_{IIC} = \int_{0}^{\Delta u^*} [\tau(\Delta u) - \tau^f] \,d(\Delta u) = (\tau^p - \tau^f) \overline{\Delta u} \) & & & $=$ \\
Ohnaka, 2003 & \( \displaystyle G_c = \int_{D_0}^{D_c} [\tau(D)-\tau_r] \,dD \) & & & $=$ \\
Nielsen et al., 2016a & \( \displaystyle G_f(u) = \int_{0}^{u} (\tau(u') - \tau(u)) \,du' \) & & & $G_f$ \\
Hou et al., 2012 & \( \displaystyle E_G = - \left [ \frac{\mu_p - \mu_{ss}}{ln(0.05)} \right ] D_c \sigma_n \approx 0.334 (\mu_p - \mu_{ss}) D_c \sigma_n \) & & & Wb(Dc) \\
Hirose \& Bystricky, 2007; Seyler et al., 2020 & \( \displaystyle E_g \) from Seyler et al., 2020 & & & Wb(Dc) \\
Hirose \& Bystricky, 2007; Seyler et al., 2020 & \( \displaystyle W_b\) from Seyler et al., 2020 & & & Wb(Dth) \\
Violay et al., 2013 & fracture energy defined as the integral between the friction curve and the minimum level reached (Rice, 2006) & & ** & Wb(Dc) \\
Togo et al., 2011b & \( \displaystyle E_G = - \left [ \frac{\mu_p - \mu_{ss}}{ln(0.05)} \right ] D_c \sigma_n \approx 0.334 (\mu_p - \mu_{ss}) D_c \sigma_n \) & & & Wb(Dc) \\
Cornelio et al., 2019 & See above & & & Wb(Dc) \\
Seyler et al., 2020 & \( \displaystyle E_g = \int_{0}^{D_c} (\tau_p - \tau_{ss}) e^{-\frac{\delta}{D_c}} \,d\delta \) & & ** & Wb(Dc) \\
Seyler et al., 2020 & \( \displaystyle W_b = \int_{0}^{D_{th}} (\tau_p - \tau_{ss}) e^{-\frac{\delta}{D_{th}}} \,d\delta \) & & & Wb(Dth) \\
Faulkner et al., 2011; Seyler et al., 2020 & \( \displaystyle E_g \) from Seyler et al., 2020 & & & Wb(Dc) \\
Faulkner et al., 2011; Seyler et al., 2020 & \( \displaystyle W_b \) from Seyler et al., 2020 & & & Wb(Dth) \\
Ujiie \& Tsutsumi, 2010; Seyler et al., 2020 & \( \displaystyle E_g \) from Seyler et al., 2020 & & & Wb(Dc) \\
Ujiie \& Tsutsumi, 2010; Seyler et al., 2020 & \( \displaystyle W_b \) from Seyler et al., 2020 & & & Wb(Dth) \\
Ujiie et al., 2013; Seyler et al., 2020 & \( \displaystyle E_g \) from Seyler et al., 2020 & & & Wb(Dc) \\
Ujiie et al., 2013; Seyler et al., 2020 & \( \displaystyle W_b \) from Seyler et al., 2020 & & & Wb(Dth) \\
Oohashi et al., 2015; Seyler et al., 2020 & \( \displaystyle E_g \) from Seyler et al., 2020 & & & Wb(Dc) \\
Oohashi et al., 2015; Seyler et al., 2020 & \( \displaystyle W_b \) from Seyler et al., 2020 & & & Wb(Dth) \\
Vannucchi et al., 2017 & \( \displaystyle E_r = E - G_f \) & & & not used \\
Vannucchi et al., 2017 & \( \displaystyle G_{fSND} \approx G_f\) from Nielsen et al., 2016a & & & Wb(Dc) \\
Vannucchi et al., 2017 & \( \displaystyle E = \frac{1}{2} (\tau_0 - \tau_{ss}) D \) & & & not used \\
Vannucchi et al., 2017 & \( \displaystyle G_{fREF} \approx E_g \) from Seyler et al., 2020 & & & not used \\
Vannucchi et al., 2017 & \( \displaystyle E_{REG} = E - G_{fREG} \) & & & not used \\
Vannucchi et al., 2017; Seyler et al., 2020 & \( \displaystyle W_b \) from Seyler et al., 2020 & & & Wb(Dth) \\
Boutareaud et al., 2012 & The total fracture energy is the area under the slip-weakening curve, between the peak shear stress and the residual shear stress & & & Wb(Dc) \\
Boutareaud et al., 2012 & The frictional heat is the work of the residual shear stress & & & not used \\
Boutareaud et al., 2012 & The dissipated energy is the total mechanical energy (i.e. the sum of the total fracture energy and the frictional heat) & & & not used \\
Boutareaud et al., 2012; Seyler et al., 2020 & \( \displaystyle W_b \) from Seyler et al., 2020 & & & Wb(Dth) \\
Aretusini et al., 2021a & \( \displaystyle G_f\) from Nielsen et al., 2016a, calculated from $\tau_p$ to $\tau(D_c)$ & & & Wb(Dc) \\
Sawai et al., 2014 & We calculated the fracture energy at coseismic slip velocity from the area under the shear stress versus displacement curve (i.e., from $\tau_p$ to $\tau(D_c)$) & & & Wb(Dc) \\
Sawai et al., 2014; Seyler et al., 2020 & \( \displaystyle W_b \) from Seyler et al., 2020 & & & Wb(Dth) \\
Togo \& Shimamoto, 2012 & Specific fracture energy (G or $E_G/S$) was calculated by integrating $(\tau-\tau_ss) = (\mu - \mu_{ss})\sigma_N$ with respect to displacement, from displacement at peak friction to the displacement at the ened of a run & $J/m^2$ & & Wb(Dc) \\
Togo \& Shimamoto, 2012 & Total frictional work per unit fault area ($W_F/S$) was calculated by integrating $\tau = \mu \sigma_N$ with respect to displacement from displacement at peak friction to the displacement at the end of a run & $J/m^2$ & ** & Wb(Dc) \\
Togo et al., 2016 & Specific fracture energy $E_G$ & $J/m^2$ & ** & Wb(Dc) \\
De Paola et al., 2011 & $W_b$ & & ** & Wb(Dc) \\
Boulton et al., 2017 & Specific fracture energy [...] calculated as the integral of the raw friction data multiplied by the applied normal stress (from slip of $\mu_p$ to $d_w$) & & & Wb(Dc) \\
French et al., 2014; Seyler et al., 2020 & \( \displaystyle E_g \)  Seyler et al., 2020 & & & Wb(Dc) \\
French et al., 2014; Seyler et al., 2020 & \( \displaystyle W_b \) from Seyler et al., 2020 & & & Wb(Dth) \\
Chen et al., 2017 & The breakdown work $W_b$ calculated from $\mu_{pk}$, $\mu_{ss}$ and $D_w$ determined following Togo et al., 2011 & & & Wb(Dc) \\
Mizoguchi et al., 2007 & Integral from peak friction $\mu$ to $\mu_r$ ($D_c$) of friction data best fit & $N/m$*** & & Wb(Dc) \\
Mizoguchi et al., 2007; Seyler et al., 2020 & \( \displaystyle W_b \) from Seyler et al., 2020 & $N/m$*** & & $=$ \\
Sawai et al., 2012 & Integration of \( \displaystyle \mu(d) = \mu_{ss}+(\mu_p-\mu_{ss}) e^{\frac{ln(0.05) d}{D_c}} \) & N/m*** & & Wb(Dc) \\
Sawai et al., 2012; Seyler et al., 2020 & \( \displaystyle W_b \) from Seyler et al., 2020 & & & Wb(Dth) \\
Brantut et al., 2008 & $G_c$ is the area under the slip-weakening curve, between the peak friction and the residual friction & & & Wb(Dc) \\
Brantut et al., 2008 & \( \displaystyle H\) & & & not used \\
Brantut et al., 2008 & \( \displaystyle E_{fracture}\) & & & not used \\
Brantut et al., 2008 & \( \displaystyle E_{am}\) & & & not used \\
Brantut et al., 2008 & \( \displaystyle E_{dhy}\) & & & not used \\
Brantut et al., 2008 & \( \displaystyle E_{heat}\) & & & not used \\
Brantut et al., 2008; Seyler et al., 2020 & \( \displaystyle W_b \) from Seyler et al., 2020 & & & Wb(Dth) \\
Togo et al., 2011b; Seyler et al., 2020 & \( \displaystyle W_b \) from Seyler et al., 2020 & & & Wb(Dth) \\
Smith et al., 2013 & \( \displaystyle E_g = \int_{0}^{D_c} (\tau_p - \tau_{ss}) e^{-\frac{\delta}{D_c}} \,d\delta \) & & **** & Wb(Dc) \\
Rempe et al., 2017 & \( \displaystyle E_g = \int_{0}^{D_c} (\tau_p - \tau_{ss}) e^{-\frac{\delta}{D_c}} \,d\delta \) & & **** & Wb(Dc) \\
Yao et al., 2013a & Gray area beneath the slip-weakening curve gives specific fracture energy, $E_G$, normalized with respect to the normal stress $\sigma_n$. & $N/m$*** & & Wb(Dc) \\
Aubry et al., 2018 & \( \displaystyle E_{tot} = \frac{1}{2} (\tau_0+\tau_r) d \) & $J/m^2$ & & $=$ \\
Aubry et al., 2018 & \( \displaystyle Q_{th} = 2 \.{q}_{t=0} t_w \sqrt{\pi \kappa t_w} \) & & & not used \\
Lockner \& Okubo, 1983 & \( \displaystyle W_T = \frac{1}{2} (\tau_i + \tau_e) \Delta x \) & $J/m^2$ & & $=$ \\
Lockner \& Okubo, 1983 & \( \displaystyle W_f = \tau_f u \) & & & not used \\
Lockner \& Okubo, 1983 & \( \displaystyle H_f = \sqrt{\pi} \frac{k}{\alpha} a_T T_{(x,t)} exp \left [ - \left ( \frac{x}{a_T} \right )^2 \right ] \) & & & not used \\
Scuderi et al., 2020 & $W_b$ from $\tau_{peak}$ to $\tau_{minimum}$ (at slip $D_c$) & $J/m^2$ & & $=$ \\
\hline
\end{longtable}
CD: crack density, AWT: acoustic wave tomography, LEFM: linear elastic fracture mechanics, CZM: cohesive zone model.
*: Re-calculated for this work, using the $G_f(u)$ binned in slip.
**: The definition of energy was not specified in the reference.
***: The energy was converted into units of $J/m^2$ by multiplying for the total normal stress.
****: Re-calculated for this work using the original data from the reference.

\section{Models}

\begin{longtable}{|p{3.5 cm}|p{6.5 cm}|p{1.25 cm}|p{1.25 cm}|p{1.5 cm}|}
\hline
\textbf{Reference} & \textbf{Original definition} & \textbf{Unit} & \textbf{Note} & \textbf{Name in repository} \\ \hline
Tinti et al., 2005b; Tinti et al., 2008 & \( \displaystyle W_b = \int_{0}^{T_b} (\bm{\tau}(t)-\bm{\tau}_{min})\cdot \bm{v}(t)\,dt \) & $J/m^2$ & & average Wb \\
Tinti et al., 2005b; Tinti et al., 2008 & \( \displaystyle W_r = \int_{T_b}^{T} (\bm{\tau}(t)-\bm{\tau}_{min})\cdot \bm{v}(t)\,dt \) & $J/m^2$ & & not used \\
Spagnuolo, 2006 & \( \displaystyle W_b = \int_{0}^{T_b} (\vec{\tau}(t)-\vec{\tau_{min}})\cdot \vec{\dot{u}}(t)\,dt \) & $J/m^2$ & & average Wb \\
Spagnuolo, 2006 & \( \displaystyle W_R = \int_{T_b}^{T} (\vec{\tau}(t)-\vec{\tau_{min}})\cdot \vec{\dot{u}}(t)\,dt \) & $J/m^2$ & & not used \\
Tinti et al., 2021 & \(\displaystyle E_g = G_c = \frac{1}{2} (\tau_y-\tau_f) D_c\) & $J/m^2$ & * & average Wb ($>$20\%) \\
Lambert \& Lapusta, 2020 & \( \displaystyle G = \frac{\int_{\Omega} G_{loc}(z) \,dz}{\int_{\Omega}\,dz} \) & $J/m^2$ & & average Wb \\
Lambert \& Lapusta, 2020 & \( \displaystyle G_{loc}(z) = \int_{0}^{D_c (z)} [\tau(\delta')-\tau_{min}(z)]\,d\delta' \), with $\tau(D_c(z)) = \tau_{min}(z)$ & $J/m^2$ & & not used \\
Guatteri et al., 2003 & median of $G_c$ across the fault plane & $J/m^2$ & & average Wb \\
Peyrat et al., 2001 & \( \displaystyle G=\int_{0}^{D_c} [T(D)-T_f]\,dD = \frac{1}{2}(T_u-T_f)D_c \) & $J/m^2$ & & average Wb \\
Guatteri et al., 2001 & median $G_c$ estimated from the stress-slip curve as the area beneath the weakening part & $J/m^2$ & ** & average Wb \\
Favreau \& Archuleta, 2003 & \( \displaystyle W_e^\Sigma = \int_{\Sigma} w_e^\Sigma\,d\Sigma, w_e^\Sigma = -\frac{1}{2} \delta u^1 (\tau^1-\tau^0) \) & $J/m^2$ & & not used \\
Favreau \& Archuleta, 2003 & \( \displaystyle W_f^\Sigma = \int_{\Sigma} w_f^\Sigma\,d\Sigma, w_f^\Sigma = \frac{1}{2} (\tau_s-\tau^y(\delta u^1)) inf(\delta u^1, D_c) \) & $J/m^2$ & & average Wb \\
Favreau \& Archuleta, 2003 & \( \displaystyle W_r^\Sigma = \int_{\Sigma} w_r^\Sigma\,d\Sigma, w_r^\Sigma = \delta u^1 (\tau^y (\delta u^1)-\tau^1) \) & $J/m^2$ & & not used \\
Gallovic et al., 2019b & \( \displaystyle G_b =\frac{1}{2}(\tau_s-\tau_d)D_c  \) & $J/m^2$ & & average Wb \\
\hline
\end{longtable}
Average (or mean) here means the spatial average over the whole fault area. *: averaged over fault area with slip larger than 20\% of average slip. **: similar results using $D_c$ and Rate and State Friction $d_c$.

\section{Seismology}

\begin{longtable}{|p{3.5 cm}|p{6.5 cm}|p{1.25 cm}|p{1.25 cm}|p{1.5 cm}|}
\hline
\textbf{Reference} & \textbf{Original definition} & \textbf{Unit} & \textbf{Note} & \textbf{Name in repository} \\ \hline
All references & \( \displaystyle \delta= \frac{M_0}{\mu \pi r^2} \) & m & * & slip \\
All references & \( \displaystyle G'= \frac{1}{2}(\Delta \tau - 2\tau_a)\delta \) & $J/m^2$ & *  & G' \\
All references & \( \displaystyle \tau_a = \mu \frac{E_s}{M_0} \) & Pa & * & not used \\
Viesca \& Garagash, 2015 & \( \displaystyle G^{max}=G'+(\tau_i + \Delta \tau)\delta \) & $J/m^2$ & & Gmax \\

\hline
\end{longtable}
*: Re-calculated for this work. For laboratory seismic events in Selvadurai, (2019) we used the shear modulus $\mu$ of PMMA (2.27$\cdot10^9$ Pa). For all the other seismic events, we used the shear modulus $\mu$ of granite (30$\cdot10^9$ Pa). In particular we used for Selvadurai, (2019) laboratory seismic events the stress drop and source radius estimated with Brune model, the radiated energy obtained from P waves, and the source parameters using the omega squared model.				

% \bibliographystyle{apalike}
% \bibliography{references}

\end{document}
